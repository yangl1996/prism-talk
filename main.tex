\documentclass[aspectratio=169]{beamer}
\mode<presentation>
\usetheme{default}
\usecolortheme{beaver}

\usepackage{graphicx, color, microtype}
\usepackage{tikz}
\usepackage{tikzsymbols}
\usetikzlibrary{arrows}
\usetikzlibrary{positioning}
\usetikzlibrary{decorations.pathreplacing}
\usetikzlibrary{fit}

\usepackage{pgfpages}
\setbeameroption{show notes on second screen}

\usepackage[style=authortitle,backend=biber]{biblatex}
\addbibresource{ref.bib}

\title{Prism: Scaling Bitcoin by 10,000$\times$}

\author{Lei~Yang \inst{1} \and Vivek~Bagaria \inst{2} \and Gerui~Wang \inst{3}
\and Mohammad~Alizadeh \inst{1} \and David~Tse \inst{2} \and Giulia~Fanti \inst{4} \and Pramod~Viswanath \inst{3}}
\institute{\inst{1} MIT CSAIL \and \inst{2} Stanford University \and \inst{3}
University of Illinois Urbana-Champaign \and \inst{4} Carnegie Mellon University}

\date[SBC 2020]{The Stanford Blockchain Conference 2020}

\begin{document}
\beamertemplatenavigationsymbolsempty

\begin{frame}
\titlepage
    \note{start with ``I'm excited to be here to talk about''}
\end{frame}

\begin{frame}
    \note{we all know that bitcoin has very good security...}
    \frametitle{Bitcoin: high security, low throughput, and long latency}
    \begin{block}{}
    \begin{itemize}
        \item \alert{Security}: 50\% adversary
            \pause
        \item \alert{Throughput}: 7 tps
        \item \alert{Confirmation Latency}: hours
    \end{itemize}
    \end{block}
\end{frame}

\begin{frame}
    \frametitle{Prism: decoupling performance from security}
    \note{our main finding is that bitcoin couples security and thorughput, latency. the new protocol called ``Prism'' decouples performance from security}
    \begin{itemize}
            \pause
        \item \alert{Bitcoin}: throughput, latency coupled with security
            \pause
        \item \alert{Prism}: decouple performance from security
    \end{itemize}
\end{frame}

\begin{frame}
    \frametitle{This talk}
    \tableofcontents
\end{frame}

\section{The Prism consensus protocol}

\begin{frame}
    \frametitle{This talk}
    \tableofcontents[currentsection,currentsubsection]
\end{frame}

\begin{frame}
    \note{before talking about Prism, we need to first look at Bitcoin... Define forking}
    \frametitle{Bitcoin: the longest chain rule}
    \begin{figure}
        \tikzstyle{block} = [draw, fill=blue!20, rectangle, rounded corners, minimum height=1em, minimum width=1.6em, text centered]
        \begin{tikzpicture}[auto, node distance=2em, >=latex']
            \node[block](genesis){};
            \pause
            \node[block, below of=genesis](1){};
            \draw[->](1) to (genesis);
            \pause
            \node[block, below of=1](2){};
            \draw[->](2) to (1);
            \pause
            \node[block, below of=2](3){};
            \node[block, right of=3](3fork){};
            \draw[->](3fork) to (2);
            \draw[->](3) to (2);
            \node[block, below of=3](4){};
            \draw[->](4) to (3);
            \pause
            \node[block, below of=4](5){};
            \draw[->](5) to (4);
        \end{tikzpicture}
    \end{figure}
    \uncover<+->{Mining rate = 1 block per 10 min}
\end{frame}


\begin{frame}
    \note{improve the script (1.5 min). }
    \note{another important aspect of Bitcoin is when do we confirm transactions. define confirm. attacker gets lucky mining faster than the honest party for the first few blocks. unlikely to keep ahead in the long run}
    \frametitle{Bitcoin: $k$-deep confirmation rule}
    \begin{columns}
    \column{.5\textwidth}
    \begin{figure}
        \tikzstyle{block} = [draw, fill=blue!20, rectangle, rounded corners, minimum height=1.2em, minimum width=1.9em, text centered]
        \tikzstyle{confirmed} = [draw, fill=green!20, rectangle, rounded corners, minimum height=1.2em, minimum width=1.9em, text centered]
        \tikzstyle{deconfirmed} = [draw,dashed, fill=yellow!20, rectangle, rounded corners, minimum height=1.2em, minimum width=1.9em, text centered]
        \tikzstyle{badblock} = [draw, fill=red!20, rectangle, rounded corners, minimum height=1.2em, minimum width=1.9em, text centered]
        \tikzstyle{badblockunreleased} = [draw,dashed, fill=red!10, rectangle, rounded corners, minimum height=1.2em, minimum width=1.9em, text centered]
        \begin{tikzpicture}[auto, node distance=2.5em, >=latex']
            \node<1->[block](genesis){};
            \uncover<2,9-10>{
                \node[block, below of=genesis](mine){TX};
                \draw[->](mine) to (genesis);
            }
            \uncover<3-5, 11->{
                \node[confirmed, below of=genesis](mineconfirmed){TX};
                \draw[->](mineconfirmed) to (genesis);
                \node[left=1em of mineconfirmed](){Confirmed};
            }
            \uncover<6-8>{
                \node[deconfirmed, below of=genesis](minedeconfirmed){TX};
                \draw[->](minedeconfirmed) to (genesis);
            }
            \uncover<4>{
                \node[badblockunreleased, right of=mineconfirmed](bad1hidden){};
                \node[badblockunreleased, below of=bad1hidden](bad2hidden){};
                \draw[->](bad1hidden) to (genesis);
                \draw[->](bad2hidden) to (bad1hidden);
                \node[right=1em of bad1hidden](){Private chain};
            }
            \uncover<5-8,12->{
                \node[badblock, right of=mineconfirmed](bad1){};
                \node[badblock, below of=bad1](bad2){};
                \draw[->](bad1) to (genesis);
                \draw[->](bad2) to (bad1);
            }
            \uncover<7-8>{
                \node[block, below of=bad2](bad3){};
                \draw[->](bad3) to (bad2);
                \node[right=1em of bad3](){Longest chain};
            }
            \uncover<8>{
                \node[block, below of=bad3](bad4){};
                \draw[->](bad4) to (bad3);
            }
            \uncover<10->{
                \node[block, below of=mine](good1){};
                \draw[->](good1) to (mine);
                \node[block, below of=good1](good2){};
                \draw[->](good2) to (good1);
                \node[block, below of=good2](good3){};
                \draw[->](good3) to (good2);
                \node[block, below of=good3](good4){};
                \draw[->](good4) to (good3);
            }
            \uncover<11->{
                \draw[decoration={brace,mirror,raise=15pt,amplitude=6pt},decorate](mine.north) -- node[left=21pt] {$k$} (good4.south);
            }
            \uncover<12->{
                \node[left=1em of good4](){Longest chain};
            }
        \end{tikzpicture}
    \end{figure}
    \column{.5\textwidth}
    \uncover<13->{
        \begin{block}{}
        30\% adversary power

        \uncover<14->{
            \alert{For $10^{-3}$ attack probability, wait 250 mins!}
        }

        Nakamoto's Table
        \smallskip
        {\small
        \centering
        \begin{tabular}{r | c}
            \hline
            $k$ & $\epsilon$ \\ \hline
            0 & 1.0000000 \\
            5 & 0.1773523 \\
            10 & 0.0416605 \\
            15 & 0.0101008 \\
            20 & 0.0024804 \\
            \alert<14->{25} & \alert<14->{0.0006132} \\
            30 & 0.0001522 \\
            35 & 0.0000379 \\
            40 & 0.0000095 \\
            45 & 0.0000024 \\
            50 & 0.0000006 \\ \hline
        \end{tabular}
        }
        \end{block}
    }
    \end{columns}
\end{frame}

\begin{frame}
    \frametitle{Calculating the performance of Bitcoin}
    \note{now that we know how bitcoin works, let's calculate its performance. remember to explain why forking increases}
    \pause
    \begin{block}{Throughput}
        \alert<5>{Block size} $\times$ \alert<4>{Mining rate}
    \end{block}
    \pause
    \begin{block}{Latency}
        Confirmation depth ($k$) $\div$ \alert<4>{Mining rate}
    \end{block}
\end{frame}

\begin{frame}
    \frametitle{Increasing the mining rate or the block size causes forking}
    \note{forking happens when\dots}
    \begin{columns}
        %TODO: should we add a figure here
        \column{.5\textwidth}
        Forking happens when the next block is mined before the previous one is received.
        \pause
        \column{.5\textwidth}
        \begin{itemize}
            \item Mine faster $\rightarrow$ Shorter interval
                \pause
            \item Bigger blocks $\rightarrow$ Slower propagation
        \end{itemize}
    \end{columns}
\end{frame}

\begin{frame}
    \frametitle{Forking compromises security}
    \note{why forking compromises security. then how bad is it. forking essentially reduces the honest mining power}
    \begin{columns}
        \column{.5\textwidth}
        \begin{figure}
            \tikzstyle{block} = [draw, fill=blue!20, rectangle, rounded corners, minimum height=1em, minimum width=1.6em, text centered]
            \tikzstyle{badblockunreleased} = [draw,dashed, fill=red!10, rectangle, rounded corners, minimum height=1em, minimum width=1.6em, text centered]
            \begin{tikzpicture}[auto, node distance=2em, >=latex']
                \uncover<1->{
                \node[block](genesis){};
                \node[block, below of=genesis](1){};
                \draw[->](1) to (genesis);
                \node[block, below of=1](2){};
                \draw[->](2) to (1);
                \node[block, below of=2](3){};
                \draw[->](3) to (2);
                \node[block, below of=3](4){};
                \draw[->](4) to (3);
                \node[block, below of=4](5){};
                \draw[->](5) to (4);
                }
                \uncover<3->{
                \node[badblockunreleased, right of=1](b1){};
                \draw[->](b1) to (genesis);
                \node[badblockunreleased, below of=b1](b2){};
                \draw[->](b2) to (b1);
                \node[badblockunreleased, below of=b2](b3){};
                \draw[->](b3) to (b2);
                \node[badblockunreleased, below of=b3](b4){};
                \draw[->](b4) to (b3);
                }
                \uncover<2->{
                \node[block, right=5em of genesis](genesis){};
                \node[block, below of=genesis](1){};
                \draw[->](1) to (genesis);
                \node[block, right of=1](2){};
                \draw[->](2) to (genesis);
                \node[block, below of=2](3){};
                \draw[->](3) to (2);
                \node[block, below of=1](4){};
                \draw[->](4) to (1);
                \node[block, left of=4](5){};
                \draw[->](5) to (1);
                }
                \uncover<3->{
                \node[badblockunreleased, right of=2](b1){};
                \draw[->](b1) to (genesis);
                \node[badblockunreleased, below of=b1](b2){};
                \draw[->](b2) to (b1);
                \node[badblockunreleased, below of=b2](b3){};
                \draw[->](b3) to (b2);
                \node[badblockunreleased, below of=b3](b4){};
                \draw[->](b4) to (b3);
                }
            \end{tikzpicture}
        \end{figure}
        \column{.5\textwidth}
        \uncover<4->{
    \begin{figure}
        \centering
        \includegraphics{figure-source/mining-rate-security.pdf}
    }
    \end{figure}
    \end{columns}
    \uncover<4->{
        \alert{Bitcoin performance is coupled with security!}
    }
\end{frame}

\begin{frame}
    \note{as previously mentioned, prism deconstructs the bitcoin blockchain. the main insight is that a bitcoin block serves two roles... I believe every block in that chain is valid. what role is associated to what performance}
    \frametitle{Deconstructing a block: proposing and voting}
    \begin{figure}
        \tikzstyle{block} = [draw, fill=blue!20, rectangle, rounded corners, minimum height=1.3em, minimum width=2.1em, text centered]
        \begin{tikzpicture}[auto, node distance=2.3em, >=latex']
            \node[block](genesis){};
            \node[block, below of=genesis](1){};
            \draw[->](1) to (genesis);
            \node[block, below of=1](2){};
            \draw[->](2) to (1);
            \node[block, below of=2](3){};
            \draw[->](3) to (2);
            \uncover<2->{
                \node[block, below of=3](4){TX};
                \draw[->](4) to (3);
            }
            \uncover<3->{
                \node[right=1em of 4](){Add new transactions};
            }
            \uncover<4->{
                \draw[densely dashed, ->] (4.west) to [bend left, looseness=1.2] (3.west);
                \draw[densely dashed, ->] (4.west) to [bend left, looseness=1.2] (2.west);
                \draw[densely dashed, ->] (4.west) to [bend left, looseness=1.2] (1.west);
                \draw[densely dashed, ->] (4.west) to [bend left, looseness=1.2] (genesis.west);
                \node[left=2em of 2](){Certify previous blocks};
            }
        \end{tikzpicture}
    \end{figure}
\end{frame}

\begin{frame}
    \frametitle{Scale proposing with lots of transaction blocks}
    \note{bitcoin throughput is limited because of forking. if forking does not matter anymore, we can mine as much as possible. We achieve that using transaction blcoks. then begin order}
    \note{propose tx block at high rate w/o impacting proposer forking BECAUSE proposer still mines at low rate}
    \begin{figure}
        \tikzstyle{transaction} = [draw, fill=blue!20, rectangle, rounded corners, minimum height=1em, minimum width=1.6em, text centered]
        \tikzstyle{voter} = [draw, fill=green!20, rectangle, rounded corners, minimum height=1em, minimum width=1.6em]
        \tikzstyle{proposer} = [draw, circle, fill=yellow!20]
        \begin{tikzpicture}[auto, node distance=2.3em, >=latex']
            \node[transaction](tx1){};
            \node[transaction, left of=tx1](tx2){};
            \node[transaction, left of=tx2](tx3){};
            \pause
            \node[transaction, below of=tx1](tx4){};
            \node[transaction, left of=tx4](tx5){};
            \node[transaction, left of=tx5](tx6){};
            \node[transaction, left of=tx6](tx7){};
            \pause
            \node[transaction, below of=tx4](tx8){};
            \node[transaction, left of=tx8](tx9){};
            \pause
            \node[transaction, below of=tx8](tx10){};
            \node[transaction, left of=tx10](tx11){};
            \node[transaction, left of=tx11](tx12){};
            \pause
            \node[proposer, right= 2em of tx1](p1){};
            \draw[->] (p1) to [bend right, looseness=1.0] (tx1);
            \draw[->] (p1) to [bend right, looseness=1.0] (tx2);
            \draw[->] (p1) to [bend right, looseness=1.0] (tx3);
            \pause
            \node[proposer, below of=p1](p2){};
            \draw[->] (p2) to [bend right, looseness=1.0] (tx4);
            \draw[->] (p2) to [bend right, looseness=1.0] (tx5);
            \draw[->] (p2) to [bend right, looseness=1.0] (tx6);
            \draw[->] (p2) to [bend right, looseness=1.0] (tx7);
            \draw[->] (p2) to (p1);
            \pause
            \node[proposer, below of=p2](p3){};
            \draw[->] (p3) to (p2);
            \draw[->] (p3) to [bend right, looseness=1.0] (tx8);
            \draw[->] (p3) to [bend right, looseness=1.0] (tx9);
            \pause
            \node[proposer, below of=p3](p4){};
            \draw[->] (p4) to [bend right, looseness=1.0] (tx10);
            \draw[->] (p4) to [bend right, looseness=1.0] (tx11);
            \draw[->] (p4) to [bend right, looseness=1.0] (tx12);
            \draw[->] (p4) to (p3);
        \end{tikzpicture}
    \end{figure}
    \uncover<+->{Forking does not matter among transaction blocks!}

    \uncover<+->{Throughput \cChangey[1.5]{2}, latency \cChangey[1.5]{-1}}
\end{frame}

\begin{frame}
    \note{proposer block still has two roles: propose and vote. we want to reduce the confirmation depth k here. now I'm going to introduce voter blocks. just think of it as drawing the figure in a different way. one chain? not helpful. still 25 deep. 1000 chains? 40\% chance to revert one chain, low prob to reverse half}
    \frametitle{Scale voting with lots of voter chains}
    \begin{figure}
        \tikzstyle{transaction} = [draw, fill=blue!20, rectangle, rounded corners, minimum height=1em, minimum width=1.6em, text centered]
        \tikzstyle{voter} = [draw, fill=green!20, rectangle, rounded corners, minimum height=1em, minimum width=1.6em]
        \tikzstyle{proposer} = [draw, circle, fill=yellow!20]
        \begin{tikzpicture}[auto, node distance=2.3em, >=latex']
            \node[transaction](tx1){};
            \node[transaction, left of=tx1](tx12){};
            \node[transaction, left of=tx12](tx13){};
            \node[proposer, right= 2em of tx1](p1){};
            \draw[->] (p1) to [bend right, looseness=1.0] (tx1);
            \draw[->] (p1) to [bend right, looseness=1.0] (tx12);
            \draw[->] (p1) to [bend right, looseness=1.0] (tx13);
            \node[transaction, below of=tx1](tx2){};
            \node[transaction, left of=tx2](tx22){};
            \node[transaction, left of=tx22](tx23){};
            \node[transaction, left of=tx23](tx24){};
            \node[proposer, below of=p1](p2){};
            \draw[->] (p2) to [bend right, looseness=1.0] (tx2);
            \draw[->] (p2) to [bend right, looseness=1.0] (tx22);
            \draw[->] (p2) to [bend right, looseness=1.0] (tx23);
            \draw[->] (p2) to [bend right, looseness=1.0] (tx24);
            \uncover<1>{
                \draw[->] (p2) to (p1);
            }
            \node[transaction, below of=tx2](tx3){};
            \node[transaction, left of=tx3](tx32){};
            \node[proposer, below of=p2](p3){};
            \uncover<1>{
                \draw[->] (p3) to (p2);
            }
            \draw[->] (p3) to [bend right, looseness=1.0] (tx3);
            \draw[->] (p3) to [bend right, looseness=1.0] (tx32);
            \node[transaction, below of=tx3](tx4){};
            \node[transaction, left of=tx4](tx41){};
            \node[transaction, left of=tx41](tx42){};
            \node[proposer, below of=p3](p4){};
            \draw[->] (p4) to [bend right, looseness=1.0] (tx4);
            \uncover<1>{
                \draw[->] (p4) to (p3);
            }
            \uncover<3->{
                \node[voter, right=2em of p1](v11){};
                \node[voter, below of=v11](v12){};
                \node[voter, below of=v12](v13){};
                \node[voter, below of=v13](v14){};
                \draw[->] (v12) to (v11);
                \draw[->] (v13) to (v12);
                \draw[->] (v14) to (v13);
                \draw[densely dashed, ->] (v11) to [bend right, looseness=1.0] (p1);
                \draw[densely dashed, ->] (v12) to [bend right, looseness=1.0] (p2);
                \draw[densely dashed, ->] (v13) to [bend right, looseness=1.0] (p3);
                \draw[densely dashed, ->] (v14) to [bend right, looseness=1.0] (p4);
            }
            \uncover<4->{
                \node[voter, right of=v11](v21){};
                \node[voter, below of=v21](v22){};
                \node[voter, below of=v22](v23){};
                \node[voter, below of=v23](v24){};
                \draw[->] (v22) to (v21);
                \draw[->] (v23) to (v22);
                \draw[->] (v24) to (v23);
                \draw[densely dashed, ->] (v21) to [bend right, looseness=1.0] (p1);
                \draw[densely dashed, ->] (v22) to [bend right, looseness=1.0] (p2);
                \draw[densely dashed, ->] (v23) to [bend right, looseness=1.0] (p3);
                \draw[densely dashed, ->] (v24) to [bend right, looseness=1.0] (p4);
                \node[right of=v21](v31){\dots};
                \node[right of=v22](v32){\dots};
                \node[right of=v23](v33){\dots};
                \node[right of=v24](v34){\dots};
                \node[voter, right of=v31](v41){};
                \node[voter, below of=v41](v42){};
                \node[voter, below of=v42](v43){};
                \node[voter, below of=v43](v44){};
                \draw[->] (v42) to (v41);
                \draw[->] (v43) to (v42);
                \draw[->] (v44) to (v43);
                \draw[densely dashed, ->] (v41) to [bend right, looseness=1.0] (p1);
                \draw[densely dashed, ->] (v42) to [bend right, looseness=1.0] (p2);
                \draw[densely dashed, ->] (v43) to [bend right, looseness=1.0] (p3);
                \draw[densely dashed, ->] (v44) to [bend right, looseness=1.0] (p4);
            }
        \end{tikzpicture}
    \end{figure}
    \begin{itemize}
        \item<3-> 1 voter chain: 25-deep \cChangey[1.5]{-1}
        \item<4-> 1000 voter chains: 2-deep \cChangey[1.5]{2}
    \end{itemize}
\end{frame}

\begin{frame}
    \frametitle{Prism is provably fast and secure}
    Adversary power $\beta < 0.5$

    Network bandwidth $C$, network latency $D$, confirmation reliability $\epsilon$, $m$ voter chains

    \begin{itemize}
        \item \alert{Security}: consistency and liveness
            \pause
        \item \alert{Throughput}: $(1-\beta)C$
            \pause
        \item \alert{Confirmation Latency}: $O\left(D\right) + O\left(\frac{-D\log\left(\epsilon\right)}{m}\right)$
    \end{itemize}
\end{frame}

\begin{frame}
    \note{now that we have a good theory}
    \frametitle{Turning theory into real system}
    \begin{itemize}
            \pause
        \item 1 Gbps network $\rightarrow$ 400 Ktps?
            \pause
        \item Real bottleneck?
    \end{itemize}
\end{frame}

\section{System implementation}

\begin{frame}
    \frametitle{This talk}
    \tableofcontents[currentsection,currentsubsection]
\end{frame}

\begin{frame}
    \frametitle{Implementing Prism in Rust}
    \begin{itemize}
        \item 10,000 lines of Rust
        \item Unspent Transaction Output (UTXO) model
        \item Pay-to-public-key transactions
        \item Code available at t.leiy.me/prism-code
    \end{itemize}
\end{frame}

\begin{frame}
    \frametitle{Prism: $\sim$70,000 tps and $\sim$10s latency}
    \note{I'll cover the comparison later}
    \begin{figure}
        \centering
        \includegraphics{figure-source/compare-fig-1-4.pdf}
    \end{figure}
\end{frame}

\begin{frame}
    \frametitle{Blockchain client: consensus and ledger keeping}
    \note{say that many clients just do ordering and the performance numbers
    are not real. you have to order and execute fast}
    \begin{itemize}
        \item \alert{Consensus}: keep track of the blockchains and mine
        \item \alert{Ledger keeping}: process transactions and calculate the UTXO set
    \end{itemize}
    \pause
    \alert{Must be fast at both}
\end{frame}

\begin{frame}
    \frametitle{What we learned}
        \begin{itemize}
            \item Bottlenecks: ledger keeping, SSD
            \item Must parallelize
        \end{itemize}
\end{frame}

\begin{frame}
    \frametitle{Parallelize ledger keeping with scoreboarding}
    \begin{figure}
        \tikzstyle{transaction} = [draw, fill=blue!20, rectangle, minimum height=1em, minimum width=1.6em, text centered]
        \tikzstyle{cpu} = [draw, densely dashed, rectangle, inner sep=1.5mm, line width=1.8pt, color=olive]
        \tikzstyle{coin} = [draw, circle, fill=yellow!20, text centered]
        \tikzstyle{badcoin} = [draw, circle, fill=red!20, text centered]
        \begin{tikzpicture}[auto, node distance=2.2em, >=latex']
            \node [transaction](t1){A$\rightarrow$C};
            \node [transaction, below of=t1](t2){A$\rightarrow$D};
            \node [transaction, below of=t2](t3){B$\rightarrow$E};
            \node [transaction, below of=t3](t4){E$\rightarrow$F};
            \uncover<1-5,10-13,16-19>{
                \node [coin,left=2em of t1](b){B};
            }
            \uncover<1-2,10-11,16-19>{
                \node [coin,left=1em of b](a){A};
            }
            \uncover<3-9,20->{
                \node [coin, left=1em of b](c){C};
            }
            \uncover<6-7,20-23>{
                \node [coin, left=2em of t1](e){E};
            }
            \uncover<8-9,24->{
                \node [coin, left=2em of t1](f){F};
            }
            \uncover<12-15>{
                \node [badcoin, left=1em of b](d){D};
            }
            \uncover<14-15>{
                \node [badcoin, left=2em of t1](e2){E};
            }

            \uncover<2-3,11-12,18-20>{
                \node [cpu, fit=(t1)]{};
            }
            \uncover<4,11-12,23-24>{
                \node [cpu, fit=(t2)]{};
            }
            \uncover<5-6,13-14,19-20>{
                \node [cpu, fit=(t3)]{};
            }
            \uncover<7-8,13-14,23-24>{
                \node [cpu, fit=(t4)]{};
            }
        \end{tikzpicture}
    \end{figure}
    \uncover<16->{
        \alert{Scoreboard: }
        \only<17-18>{
            A, C
        }
        \only<19-20>{
            A, B, C, E
        }
        \only<21>{}
        \only<22-24>{A, D, E, F}
        \only<25>{}
    }
\end{frame}

\begin{frame}
    \frametitle{Parallelize ledger keeping with scoreboarding}
    \begin{figure}
        \centering
        \includegraphics{figure-source/resource-fig-cpu.pdf}
    \end{figure}
\end{frame}

\begin{frame}
    \frametitle{Parallelize consensus with async ledger update}
    \begin{itemize}
        \item Ledger updates are ``infrequent''
            \pause
        \item Don't do it every new block
    \end{itemize}
\end{frame}

\begin{frame}
    \frametitle{Reduce spams using random jittering}
    \begin{figure}
        \tikzstyle{node} = [draw, fill=green!20, rectangle, 
        minimum height=4em, minimum width=6em, text centered, text width=5em]
        \tikzstyle{transaction} = [draw, fill=yellow!20, rectangle, minimum height=1em, minimum width=1.6em, text centered]
        \tikzstyle{block} = [draw, fill=blue!20, rectangle, rounded corners, minimum height=2.5em, minimum width=2.5em, text centered]
        \begin{tikzpicture}[auto, node distance=4em,>=latex']
            \node[node](n1){Node 1};
            \node[node, right=7em of n1](n2){Node 2};
            \draw[<->, line width=1.8pt](n1) -- (n2);
            \uncover<2,5-7>{
                \node[transaction, below of=n1](t1){A$\rightarrow$B};
            }
            \uncover<6-7>{
                \node[left=1em of t1]{Delayed};
            }
            \uncover<2,5>{
                \node[transaction, below of=n2](t2){A$\rightarrow$C};
            }
            \uncover<3-4>{
                \node[block, above of=n1](b1){A$\rightarrow$B};
                \node[block, above of=n2](b2){A$\rightarrow$C};
            }
            \uncover<4>{
                \node[block, right= 0.5em of b1](b21){A$\rightarrow$C};
                \node[block, left= 0.5em of b2](b12){A$\rightarrow$B};
            }
            \uncover<6-8>{
                \node[block, above of=n2](b2){A$\rightarrow$C};
            }
            \uncover<7-8>{
                \node[block, above of=n1](b21){A$\rightarrow$C};
            }

        \end{tikzpicture}
    \end{figure}
\end{frame}

\begin{frame}
    \frametitle{Reduce spams using random jittering}
    \begin{figure}
        \centering
        \includegraphics{figure-source/attack-fig-spamming.pdf}
    \end{figure}
\end{frame}

\section{Evaluation results and findings}


\begin{frame}
    \tableofcontents[currentsection,currentsubsection]
\end{frame}

\begin{frame}
    \frametitle{Testbed setup}
    \begin{itemize}
        \item 100 - 1000 AWS EC2 instances
        \item Random 4-regular graph
        \item 120ms propagation delay
        \item 400 Mbps bandwidth
    \end{itemize}
\end{frame}

\begin{frame}
    \frametitle{Experiments}
    \note{explain the experiments not just read the bullet points}
        \begin{itemize}
            \item \alert<2>{Comparison of throughput, latency with Algorand, Bitcoin-NG, Longest Chain}
            \item Scalability to 1000 nodes
            \item \alert<2>{CPU, bandwidth utilization}
            \item Censorship, spamming, balancing attacks
        \end{itemize}
\end{frame}

\begin{frame}
    \frametitle{Comparison with Algorand, Bitcoin-NG, Nakamoto}
    % mention algorand bottleneck
    \only<1>{
    \begin{figure}
        \centering
        \includegraphics{figure-source/compare-fig-1-1.pdf}
    \end{figure}
    }
    \only<2>{
    \begin{figure}
        \centering
        \includegraphics{figure-source/compare-fig-1-2.pdf}
    \end{figure}
    }
    \only<3>{
    \begin{figure}
        \centering
        \includegraphics{figure-source/compare-fig-1-3.pdf}
    \end{figure}
    }
    \only<4>{
    \begin{figure}
        \centering
        \includegraphics{figure-source/compare-fig-1-4.pdf}
    \end{figure}
    }
    \only<5>{
    \begin{figure}
        \centering
        \includegraphics{figure-source/compare-fig-2.pdf}
    \end{figure}
    }
    \only<6>{
    \begin{figure}
        \centering
        \includegraphics{figure-source/compare-fig.pdf}
    \end{figure}
    }
\end{frame}

\begin{frame}
    \frametitle{Our implementation is efficient}
    \begin{block}{CPU}
    \begin{itemize}
        \item Signature check: 22\%
        \item RocksDB: 53\%
        \item Data serialization: 7\%
    \end{itemize}
    \end{block}
    \begin{block}{Bandwidth}
    \begin{itemize}
        \item Transaction blocks: 99.5\%
        \item Voter blocks: 0.4\%
        \item Proposer blocks: 0.1\%
    \end{itemize}
    \end{block}
\end{frame}

\begin{frame}
    \frametitle{Takeaways}
    \begin{block}{}
    \begin{itemize}
        \item Deconstruct blockchain into \alert{proposing} and \alert{voting}
        \item Scale each part to the physical limit
        \item Must parallelize everying in the client
        \item Current bottleneck is ledger keeping and SSD
    \end{itemize}
    \end{block}
    \begin{block}{Resources}
    \begin{itemize}
        \item Code: t.leiy.me/prism-code
        \item Theory Paper: Deconstructing the Blockchain to Approach Physical Limits
        \item System Paper: t.leiy.me/prism-paper
        \item Online Demo: t.leiy.me/prism-demo
    \end{itemize}
    \end{block}
\end{frame}

\end{document}
